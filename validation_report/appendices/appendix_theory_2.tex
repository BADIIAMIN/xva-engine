\subsection{Interpolation and discretisation: theoretical considerations}
A continuous yield curve is approximated in practice by values at a finite set of maturity pillars
$\{T_k\}_{k=0}^{K}$. Any valuation requiring intermediate maturities $T \notin \{T_k\}$ necessarily
introduces an interpolation operator $\mathcal{I}$ such that:
\[
\tilde z(t,T) = \mathcal{I}\big(\{(T_k, z(t,T_k))\}_{k=0}^{K}\big)(T).
\]
This operator imposes additional structure beyond the stochastic model specification. Different
interpolation choices correspond to different assumptions about local behaviour of discount factors,
zero rates, and implied forward rates. In general, interpolation is not arbitrage-neutral unless it is
consistent with an arbitrage-free term-structure model (e.g.\ constructed from a short-rate or HJM
specification). Consequently, discretisation and interpolation can induce artefacts such as:
(i) local curvature spikes (kinks), (ii) oscillatory implied forwards, and (iii) grid-dependent valuations.
