\section{Appendix B — Formal diagnostic definitions}
\label{app:diagnostics}

\subsection{Static discount-factor monotonicity test}

Let $\{T_k\}_{k=0}^{K}$ denote maturity pillars and let
$P^{(n)}(t_i,T_k)$ be the simulated discount factor on path $n$ at time $t_i$.

Define the indicator of a monotonicity violation on path $n$ as:
\[
\mathbb{I}^{(n)}_{i,k}
=
\mathbf{1}\!\left\{
P^{(n)}(t_i,T_{k+1}) > P^{(n)}(t_i,T_k)
\right\}.
\]

The empirical violation frequency is:
\[
\widehat{p}_{i,k}
=
\frac{1}{N}
\sum_{n=1}^{N}
\mathbb{I}^{(n)}_{i,k}.
\]

This diagnostic captures static violations of the basic no-arbitrage ordering
of discount factors across maturities.

\subsection{Kink index: cross-tenor smoothness diagnostic}

Let $z^{(n)}(t_i,T_k)$ denote the simulated zero rate at pillar $T_k$.
Define the discrete second difference:
\[
\Delta^2 z^{(n)}_{i,k}
=
z^{(n)}(t_i,T_{k+1})
-
2z^{(n)}(t_i,T_k)
+
z^{(n)}(t_i,T_{k-1}),
\quad k=1,\dots,K-1.
\]

The \emph{kink index} for path $n$ at time $t_i$ is defined as:
\[
\mathrm{Kink}^{(n)}(t_i)
=
\sum_{k=1}^{K-1}
w_k
\left|
\Delta^2 z^{(n)}_{i,k}
\right|,
\]
where $w_k$ are optional scaling weights (e.g.\ based on maturity spacing).

The kink index measures local curvature and detects abrupt slope changes between
adjacent pillars. Large values indicate reduced cross-tenor smoothness and
economically implausible curve shapes.

\subsection{Discount-factor wedge: multiplicative consistency diagnostic}

For a fixed horizon $T$ and increment $u>0$, define the theoretical identity:
\[
P(t,T+u) \equiv P(t,T)\,P(t,T,T+u).
\]

Let $\widehat{P}^{(n)}(t;T,T+u)$ denote the forward discount factor reconstructed
from simulated quantities (e.g.\ via zero-rate interpolation).

The \emph{discount-factor wedge} is defined as:
\[
\mathrm{Wedge}^{(n)}(t;T,u)
=
\log P^{(n)}(t,T+u)
-
\log P^{(n)}(t,T)
-
\log \widehat{P}^{(n)}(t;T,T+u).
\]

Under pathwise arbitrage-free dynamics, the wedge should be identically zero
(up to numerical error). Persistent dispersion or maturity-dependent growth
of the wedge indicates structural inconsistency.

\subsection{Interpretation and complementarity}

\begin{itemize}
	\item DF monotonicity tests static arbitrage constraints.
	\item The kink index captures cross-sectional smoothness and local coherence.
	\item The wedge diagnostic targets dynamic multiplicative consistency.
\end{itemize}

Together, these diagnostics provide comprehensive coverage of pathwise
arbitrage-free properties.
