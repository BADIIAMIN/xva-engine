\documentclass[11pt,a4paper]{article}

% ----------------------------
% Packages
% ----------------------------
\usepackage[a4paper,margin=1in]{geometry}
\usepackage{lmodern}
\usepackage[T1]{fontenc}
\usepackage{microtype}
\usepackage{setspace}
\usepackage{parskip}

\usepackage{amsmath,amssymb,amsthm}
\usepackage{booktabs}
\usepackage{array}
\usepackage{enumitem}
\usepackage{graphicx}
\usepackage{caption}
\usepackage{subcaption}
\usepackage{float}
\usepackage{xcolor}
\usepackage{hyperref}
\usepackage[nameinlink,noabbrev]{cleveref}

% Optional, if you use UML
% \usepackage{plantuml}

% ----------------------------
% Hyperref setup
% ----------------------------
\hypersetup{
	colorlinks=true,
	linkcolor=blue,
	citecolor=blue,
	urlcolor=blue
}

% ----------------------------
% Project metadata
% ----------------------------
\title{IR Simulation Model Validation Notes \\ \large Findings  Term-Structure Diagnostics}
\author{Amine Badii}


% ----------------------------
% Paths for figures
% ----------------------------
% Adjust if your figures are elsewhere.
% If your images are in ../outputs/... relative to a section file, prefer to standardise.
\graphicspath{ {figures/finding1/}}

\begin{document}
	\maketitle
	\tableofcontents
	\newpage
	
	% ==========================================================
	% Main Body
	% ==========================================================
	
	\section{Scope and objective}
	This document substantiates key validation findings for the interest-rate simulation
	framework used for PFE, with emphasis on pathwise arbitrage diagnostics and
	benchmarking against an arbitrage-free reference model (HW1F).
	
	% --- Include Finding 1 section ---
	% Put your rewritten Finding 1 into sections/finding1.tex (recommended),
	% or change the path below accordingly.
	\section{Finding 1 — Absence of Pathwise Arbitrage-Free Term-Structure Dynamics}
\label{sec:finding1}

\subsection{Finding statement}
The IR simulation model does not enforce arbitrage-free term-structure relationships
along individual simulation paths. Zero rates at each maturity pillar are simulated
independently, with cross-tenor dependence introduced solely through correlations
between latent stochastic drivers. As a result, static and dynamic no-arbitrage
identities governing discount factors and forward-rate reconstruction may be violated
along individual paths, particularly at long horizons and under elevated volatility
conditions.

\subsection{Model features giving rise to the finding}
The finding arises directly from the structural design of the model:
\begin{itemize}[leftmargin=*]
	\item Each maturity pillar is driven by a dedicated stochastic process calibrated
	independently to market-implied variance targets.
	\item Cross-tenor dependence is introduced via correlations between drivers rather
	than through a unified term-structure factor model.
	\item No short-rate representation or arbitrage-free drift restriction (e.g.\ HJM
	consistency) is imposed.
\end{itemize}

While this construction ensures marginal calibration accuracy at each tenor, it does
not guarantee global coherence of the simulated yield curve along individual paths.

\subsection{Risk implications for PFE}
The absence of arbitrage-free dynamics may lead to:
\begin{itemize}[leftmargin=*]
	\item Violations of static discount-factor monotonicity across maturities.
	\item Reduced cross-tenor smoothness, resulting in locally implausible curve shapes.
	\item Inconsistencies in discount-factor reconstruction over time, which may
	accumulate with maturity and affect long-dated exposure profiles.
\end{itemize}
These effects are expected to be more pronounced for long-horizon PFE metrics and
portfolios with significant sensitivity to the shape of the yield curve.

\subsection{Validation strategy and diagnostics}
Three complementary diagnostics are implemented:
\begin{enumerate}[leftmargin=*]
	\item \textbf{Static discount-factor monotonicity} across maturities.
	\item \textbf{Cross-tenor smoothness} using a kink index (local curvature proxy).
	\item \textbf{Pathwise discount-factor multiplicative consistency} via a wedge metric.
\end{enumerate}
All diagnostics are benchmarked against an arbitrage-free Hull--White one-factor (HW1F)
model. Together, these diagnostics provide complementary coverage of static,
cross-sectional, and dynamic arbitrage constraints.

\subsection{Empirical evidence and plots}

\paragraph{Discount-factor monotonicity violations.}
\begin{figure}[H]
	\centering
	\includegraphics[width=0.85\textwidth]{current_df_monotonicity_heatmap.png}
	\caption{Current model: frequency of discount-factor monotonicity violations across time and pillar intervals.}
	\label{fig:finding1_df_monotonicity_current}
\end{figure}

\begin{figure}[H]
	\centering
	\includegraphics[width=0.85\textwidth]{hw_df_monotonicity_heatmap.png}
	\caption{HW1F benchmark: frequency of discount-factor monotonicity violations across time and pillar intervals.}
	\label{fig:finding1_df_monotonicity_hw}
\end{figure}

\paragraph{Kink index bands (cross-tenor smoothness).}
\begin{figure}[H]
	\centering
	\includegraphics[width=0.85\textwidth]{current_kink_bands.png}
	\caption{Current model: kink index bands over time (median and dispersion band).}
	\label{fig:finding1_kink_bands_current}
\end{figure}

\begin{figure}[H]
	\centering
	\includegraphics[width=0.85\textwidth]{hw_kink_bands.png}
	\caption{HW1F benchmark: kink index bands over time (median and dispersion band).}
	\label{fig:finding1_kink_bands_hw}
\end{figure}

\paragraph{Discount-factor wedge (multiplicative consistency).}
\begin{figure}[H]
	\centering
	\includegraphics[width=0.85\textwidth]{current_wedge_hist.png}
	\caption{Current model: wedge histogram for a representative $(T,u)$ pair.}
	\label{fig:finding1_wedge_hist_current}
\end{figure}

\begin{figure}[H]
	\centering
	\includegraphics[width=0.85\textwidth]{hw_wedge_hist.png}
	\caption{HW1F benchmark: wedge histogram for a representative $(T,u)$ pair.}
	\label{fig:finding1_wedge_hist_hw}
\end{figure}

\begin{figure}[H]
	\centering
	\includegraphics[width=0.85\textwidth]{current_wedge_vs_maturity.png}
	\caption{Current model: $p95(|\mathrm{wedge}|)$ vs maturity.}
	\label{fig:finding1_wedge_vs_maturity_current}
\end{figure}

\begin{figure}[H]
	\centering
	\includegraphics[width=0.85\textwidth]{hw_wedge_vs_maturity.png}
	\caption{HW1F benchmark: $p95(|\mathrm{wedge}|)$ vs maturity.}
	\label{fig:finding1_wedge_vs_maturity_hw}
\end{figure}

\subsection{Conclusion}
The diagnostics demonstrate that the IR simulation model does not enforce pathwise
arbitrage-free term-structure dynamics. Violations are systematic and maturity-dependent,
and are materially larger than those observed under an arbitrage-free benchmark.
This behaviour is a structural consequence of the model design and must be monitored,
benchmarked, and appropriately controlled when assessing long-horizon PFE metrics,
particularly for long-dated exposure profiles.

	
	% ==========================================================
	% Appendices
	% ==========================================================
	
	\newpage
	\appendix
	
	\section{Appendix A — Theory background}
\label{app:theory}

\subsection{No-arbitrage relationships used in diagnostics}

\paragraph{Discount factors and monotonicity.}
Let $P(t,T)$ denote the discount factor at time $t$ for maturity $T$.
In the absence of negative rates (or under standard admissibility conditions), discount factors
are non-increasing in maturity:
\[
P(t,T_{k+1}) \le P(t,T_k) \quad \text{for } T_{k+1} > T_k.
\]
Equivalently, using zero rates $z(t,T)$ defined by $P(t,T)=\exp(-z(t,T)(T-t))$,
monotonicity may be violated when simulated cross-maturity shapes become inconsistent.

\paragraph{Forward-rate reconstruction identity.}
For $0 \le t < T < T+u$, the forward discount factor satisfies
\[
P(t,T,T+u) := \frac{P(t,T+u)}{P(t,T)}.
\]
Under coherent term-structure dynamics, pricing identities ensure internal consistency of
ratios across maturity intervals.

\subsection{Pathwise vs marginal consistency}

The model may be \emph{marginally consistent} by construction (each maturity pillar calibrated
to variance targets), yet not \emph{pathwise consistent} across maturities because:
\begin{itemize}[leftmargin=*]
	\item the joint evolution across maturities is not derived from a single arbitrage-free
	term-structure model;
	\item correlations are imposed at driver level rather than induced by a coherent curve model;
	\item no drift restriction enforces HJM-consistency.
\end{itemize}

\subsection{Benchmark rationale: Hull--White one-factor (HW1F)}

HW1F provides a standard arbitrage-free reference with an explicit short-rate representation.
In the benchmark, discount factors and forward ratios inherit consistency properties from the
model’s term-structure construction (up to numerical error). Therefore, systematic deviations
under the current model can be attributed to structural design rather than plotting or discretization.

	\section{Appendix B — Formal diagnostic definitions}
\label{app:diagnostics}

\subsection{Static discount-factor monotonicity test}

Let $\{T_k\}_{k=0}^{K}$ denote maturity pillars and let
$P^{(n)}(t_i,T_k)$ be the simulated discount factor on path $n$ at time $t_i$.

Define the indicator of a monotonicity violation on path $n$ as:
\[
\mathbb{I}^{(n)}_{i,k}
=
\mathbf{1}\!\left\{
P^{(n)}(t_i,T_{k+1}) > P^{(n)}(t_i,T_k)
\right\}.
\]

The empirical violation frequency is:
\[
\widehat{p}_{i,k}
=
\frac{1}{N}
\sum_{n=1}^{N}
\mathbb{I}^{(n)}_{i,k}.
\]

This diagnostic captures static violations of the basic no-arbitrage ordering
of discount factors across maturities.

\subsection{Kink index: cross-tenor smoothness diagnostic}

Let $z^{(n)}(t_i,T_k)$ denote the simulated zero rate at pillar $T_k$.
Define the discrete second difference:
\[
\Delta^2 z^{(n)}_{i,k}
=
z^{(n)}(t_i,T_{k+1})
-
2z^{(n)}(t_i,T_k)
+
z^{(n)}(t_i,T_{k-1}),
\quad k=1,\dots,K-1.
\]

The \emph{kink index} for path $n$ at time $t_i$ is defined as:
\[
\mathrm{Kink}^{(n)}(t_i)
=
\sum_{k=1}^{K-1}
w_k
\left|
\Delta^2 z^{(n)}_{i,k}
\right|,
\]
where $w_k$ are optional scaling weights (e.g.\ based on maturity spacing).

The kink index measures local curvature and detects abrupt slope changes between
adjacent pillars. Large values indicate reduced cross-tenor smoothness and
economically implausible curve shapes.

\subsection{Discount-factor wedge: multiplicative consistency diagnostic}

For a fixed horizon $T$ and increment $u>0$, define the theoretical identity:
\[
P(t,T+u) \equiv P(t,T)\,P(t,T,T+u).
\]

Let $\widehat{P}^{(n)}(t;T,T+u)$ denote the forward discount factor reconstructed
from simulated quantities (e.g.\ via zero-rate interpolation).

The \emph{discount-factor wedge} is defined as:
\[
\mathrm{Wedge}^{(n)}(t;T,u)
=
\log P^{(n)}(t,T+u)
-
\log P^{(n)}(t,T)
-
\log \widehat{P}^{(n)}(t;T,T+u).
\]

Under pathwise arbitrage-free dynamics, the wedge should be identically zero
(up to numerical error). Persistent dispersion or maturity-dependent growth
of the wedge indicates structural inconsistency.

\subsection{Interpretation and complementarity}

\begin{itemize}
	\item DF monotonicity tests static arbitrage constraints.
	\item The kink index captures cross-sectional smoothness and local coherence.
	\item The wedge diagnostic targets dynamic multiplicative consistency.
\end{itemize}

Together, these diagnostics provide comprehensive coverage of pathwise
arbitrage-free properties.

	\section{Appendix C — Algorithmic implementation and benchmark model}
\label{app:algorithms}

\subsection{Hull--White one-factor (HW1F) model}

The Hull--White one-factor model specifies the short rate as:
\[
r(t) = x(t) + \phi(t),
\]
where $x(t)$ follows an Ornstein--Uhlenbeck process:
\[
dx(t) = -a\,x(t)\,dt + \sigma\,dW_t.
\]

The deterministic shift $\phi(t)$ is chosen to fit the initial term structure exactly.

\subsubsection{Zero-coupon bond prices}

Under HW1F, the zero-coupon bond price admits the closed form:
\[
P(t,T)
=
A(t,T)\exp\!\left(-B(t,T)\,r(t)\right),
\]
with:
\[
B(t,T) = \frac{1 - e^{-a(T-t)}}{a},
\]
and $A(t,T)$ determined to match the initial yield curve.

\subsubsection{Calibration}

Model parameters $(a,\sigma)$ are calibrated by minimizing the discrepancy between
model-implied and market-observed swaption volatilities:
\[
\min_{a,\sigma}
\sum_{i}
\left(
\sigma_{\text{model}}^{\text{swaption}}(i)
-
\sigma_{\text{market}}^{\text{swaption}}(i)
\right)^2.
\]

HW1F is arbitrage-free by construction and therefore serves as a suitable benchmark
for structural diagnostics.

\subsection{Algorithmic computation of diagnostics}

\subsubsection{DF monotonicity heatmap}

For each simulation time and maturity interval:
\begin{verbatim}
	for time i:
	for maturity k:
	violations = mean(DF[:,i,k+1] > DF[:,i,k])
\end{verbatim}

\subsubsection{Kink index bands}

\begin{verbatim}
	for time i:
	for path n:
	kink[n] = sum_k |Z[n,i,k+1] - 2 Z[n,i,k] + Z[n,i,k-1]|
	compute median and quantiles of kink
\end{verbatim}

\subsubsection{Wedge statistics}

\begin{verbatim}
	for maturity T:
	for path n:
	wedge = log(P(T+u)) - log(P(T)) - log(P_hat(T,T+u))
	compute histogram and p95(|wedge|)
\end{verbatim}

\subsection{Reproducibility and governance considerations}

All diagnostics are computed under fixed random seeds and deterministic
configuration files. Output artefacts are versioned by model configuration,
ensuring traceability between reported figures and underlying simulations.

	
\end{document}
