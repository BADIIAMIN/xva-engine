\section{Finding 2 — Pillar Discretisation and Interpolation Artefacts}
\label{sec:finding2}

\subsection{Finding statement}
The IR simulation framework represents the yield curve on a finite set of maturity pillars and
relies on numerical interpolation to obtain curve values at non-pillar maturities required for
pricing, discounting, and exposure computations. This representation introduces structural artefacts
that may affect local curve smoothness, implied forward rates, and exposure profiles. In particular,
model outputs may exhibit sensitivity to (i) the chosen interpolation scheme and (ii) the pillar grid
density.

\subsection{Model features giving rise to the finding}
The finding arises from the following design features:
\begin{itemize}
	\item Zero rates are simulated only at discrete maturity pillars $\{T_k\}$.
	\item Discount factors and forward rates at intermediate maturities are obtained via interpolation.
	\item Interpolation is applied ex post and is not enforced by the stochastic model dynamics.
\end{itemize}
Consequently, the interpolation operator imposes additional structure that is not implied by the
underlying stochastic specification.

\subsection{Risk implications for PFE}
Interpolation artefacts may lead to:
\begin{itemize}
	\item Artificial oscillations in the implied forward curve and local forward-rate spikes.
	\item Distortions of intermediate discount factors and local curve curvature.
	\item Sensitivity of exposure profiles (especially long-dated or convexity-sensitive products) to
	numerical choices rather than economic risk-factor dynamics.
\end{itemize}

\subsection{Validation strategy and diagnostics}
Three complementary diagnostics are implemented:
\begin{enumerate}
	\item \textbf{Interpolation sensitivity (T2.1):} compare interpolated curves generated from identical pillar data
	under alternative interpolation schemes (linear-on-zero vs linear-on-logDF) and report RMS / max deviations.
	\item \textbf{Implied forward roughness (T2.2):} compute the implied instantaneous forward curve on a dense grid and
	quantify roughness via an integrated curvature proxy.
	\item \textbf{Pillar density stress (T2.3):} compare curves reconstructed from the full pillar set to curves reconstructed
	from a coarse pillar subset and quantify deviations on a common dense grid.
\end{enumerate}
Diagnostics are produced for the current model and for the HW1F benchmark to separate structural interpolation artefacts
from numerical noise.

\subsection{Empirical evidence}
\begin{figure}[H]
	\centering
	\includegraphics[width=0.85\textwidth]{finding2/current_interp_sensitivity_rms_bands.png}
	\caption{Current model (T2.1): interpolation sensitivity (RMS difference) over time.}
	\label{fig:f2_interp_rms_current}
\end{figure}

\begin{figure}[H]
	\centering
	\includegraphics[width=0.85\textwidth]{finding2/hw1f_interp_sensitivity_rms_bands.png}
	\caption{HW1F benchmark (T2.1): interpolation sensitivity (RMS difference) over time.}
	\label{fig:f2_interp_rms_hw}
\end{figure}

\begin{figure}[H]
	\centering
	\includegraphics[width=0.85\textwidth]{finding2/current_forward_roughness_linear_bands.png}
	\caption{Current model (T2.2): implied forward roughness (linear-on-zero) over time.}
	\label{fig:f2_rough_lin_current}
\end{figure}

\begin{figure}[H]
	\centering
	\includegraphics[width=0.85\textwidth]{finding2/current_forward_roughness_logdf_bands.png}
	\caption{Current model (T2.2): implied forward roughness (linear-on-logDF) over time.}
	\label{fig:f2_rough_logdf_current}
\end{figure}

\begin{figure}[H]
	\centering
	\includegraphics[width=0.85\textwidth]{finding2/current_pillar_density_rms_bands.png}
	\caption{Current model (T2.3): pillar density stress (full vs coarse pillar set), RMS difference over time.}
	\label{fig:f2_density_current}
\end{figure}

\subsection{Conclusion}
The discrete pillar representation combined with ex post interpolation introduces structural artefacts that are not implied
by the underlying stochastic model. The implemented diagnostics quantify sensitivity to interpolation scheme, identify
potential roughness in implied forward curves, and demonstrate grid-dependence under pillar coarsening. These effects are
model-design driven and should be controlled via (i) consistent interpolation policy, (ii) sensitivity monitoring, and
(iii) appropriate pillar grid specification for long-dated exposure calculations.
