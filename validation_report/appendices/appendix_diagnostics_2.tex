\subsection{RMS difference (interpolation sensitivity)}
Given a dense maturity grid $\{T_j\}_{j=1}^{J}$ and two interpolated curves
$\tilde z^{(1)}(t,T)$ and $\tilde z^{(2)}(t,T)$ derived from the same pillar data,
define pointwise differences:
\[
\Delta z(t,T_j) = \tilde z^{(1)}(t,T_j) - \tilde z^{(2)}(t,T_j).
\]
The root-mean-square (RMS) difference at time $t$ is:
\[
\mathrm{RMS}_z(t)
=
\sqrt{\frac{1}{J}\sum_{j=1}^{J}\big(\Delta z(t,T_j)\big)^2}.
\]
RMS provides a stable measure of the typical magnitude of interpolation discrepancies while
penalising large local deviations.

\subsection{Implied forward curve and roughness proxy}
On a dense maturity grid, the instantaneous forward curve is defined by:
\[
f(t,T) = \frac{\partial}{\partial T}\big(T\,\tilde z(t,T)\big).
\]
A roughness proxy used in the diagnostics is the integrated absolute curvature:
\[
R_f(t) = \int \left|\frac{\partial^2 f(t,T)}{\partial T^2}\right|\,dT,
\]
approximated numerically using finite differences. Elevated $R_f(t)$ indicates oscillatory or
non-smooth forward curves potentially induced by discretisation and interpolation.

\subsection{Pillar density stress metric}
Let $\tilde z^{\mathrm{full}}(t,T)$ denote the interpolated curve reconstructed from the full pillar set
and $\tilde z^{\mathrm{coarse}}(t,T)$ the curve reconstructed from a coarse pillar subset. Define:
\[
\Delta z_{\mathrm{dens}}(t,T_j) = \tilde z^{\mathrm{coarse}}(t,T_j) - \tilde z^{\mathrm{full}}(t,T_j),
\quad
\mathrm{RMS}_{\mathrm{dens}}(t)
=
\sqrt{\frac{1}{J}\sum_{j=1}^{J}\big(\Delta z_{\mathrm{dens}}(t,T_j)\big)^2}.
\]
Large values indicate grid dependence of reconstructed curves.
