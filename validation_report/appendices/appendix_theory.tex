\section{Appendix A — Theory background}
\label{app:theory}

\subsection{No-arbitrage relationships used in diagnostics}

\paragraph{Discount factors and monotonicity.}
Let $P(t,T)$ denote the discount factor at time $t$ for maturity $T$.
In the absence of negative rates (or under standard admissibility conditions), discount factors
are non-increasing in maturity:
\[
P(t,T_{k+1}) \le P(t,T_k) \quad \text{for } T_{k+1} > T_k.
\]
Equivalently, using zero rates $z(t,T)$ defined by $P(t,T)=\exp(-z(t,T)(T-t))$,
monotonicity may be violated when simulated cross-maturity shapes become inconsistent.

\paragraph{Forward-rate reconstruction identity.}
For $0 \le t < T < T+u$, the forward discount factor satisfies
\[
P(t,T,T+u) := \frac{P(t,T+u)}{P(t,T)}.
\]
Under coherent term-structure dynamics, pricing identities ensure internal consistency of
ratios across maturity intervals.

\subsection{Pathwise vs marginal consistency}

The model may be \emph{marginally consistent} by construction (each maturity pillar calibrated
to variance targets), yet not \emph{pathwise consistent} across maturities because:
\begin{itemize}[leftmargin=*]
	\item the joint evolution across maturities is not derived from a single arbitrage-free
	term-structure model;
	\item correlations are imposed at driver level rather than induced by a coherent curve model;
	\item no drift restriction enforces HJM-consistency.
\end{itemize}

\subsection{Benchmark rationale: Hull--White one-factor (HW1F)}

HW1F provides a standard arbitrage-free reference with an explicit short-rate representation.
In the benchmark, discount factors and forward ratios inherit consistency properties from the
model’s term-structure construction (up to numerical error). Therefore, systematic deviations
under the current model can be attributed to structural design rather than plotting or discretization.
