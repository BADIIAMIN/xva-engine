\section{Finding 1 — Absence of Pathwise Arbitrage-Free Term-Structure Dynamics}
\label{sec:finding1}

\subsection{Finding statement}
The IR simulation model does not enforce arbitrage-free term-structure relationships
along individual simulation paths. Zero rates at each maturity pillar are simulated
independently, with cross-tenor dependence introduced solely through correlations
between latent stochastic drivers. As a result, static and dynamic no-arbitrage
identities governing discount factors and forward-rate reconstruction may be violated
along individual paths, particularly at long horizons and under elevated volatility
conditions.

\subsection{Model features giving rise to the finding}
The finding arises directly from the structural design of the model:
\begin{itemize}[leftmargin=*]
	\item Each maturity pillar is driven by a dedicated stochastic process calibrated
	independently to market-implied variance targets.
	\item Cross-tenor dependence is introduced via correlations between drivers rather
	than through a unified term-structure factor model.
	\item No short-rate representation or arbitrage-free drift restriction (e.g.\ HJM
	consistency) is imposed.
\end{itemize}

While this construction ensures marginal calibration accuracy at each tenor, it does
not guarantee global coherence of the simulated yield curve along individual paths.

\subsection{Risk implications for PFE}
The absence of arbitrage-free dynamics may lead to:
\begin{itemize}[leftmargin=*]
	\item Violations of static discount-factor monotonicity across maturities.
	\item Reduced cross-tenor smoothness, resulting in locally implausible curve shapes.
	\item Inconsistencies in discount-factor reconstruction over time, which may
	accumulate with maturity and affect long-dated exposure profiles.
\end{itemize}
These effects are expected to be more pronounced for long-horizon PFE metrics and
portfolios with significant sensitivity to the shape of the yield curve.

\subsection{Validation strategy and diagnostics}
Three complementary diagnostics are implemented:
\begin{enumerate}[leftmargin=*]
	\item \textbf{Static discount-factor monotonicity} across maturities.
	\item \textbf{Cross-tenor smoothness} using a kink index (local curvature proxy).
	\item \textbf{Pathwise discount-factor multiplicative consistency} via a wedge metric.
\end{enumerate}
All diagnostics are benchmarked against an arbitrage-free Hull--White one-factor (HW1F)
model. Together, these diagnostics provide complementary coverage of static,
cross-sectional, and dynamic arbitrage constraints.

\subsection{Empirical evidence and plots}

\paragraph{Discount-factor monotonicity violations.}
\begin{figure}[H]
	\centering
	\includegraphics[width=0.85\textwidth]{current_df_monotonicity_heatmap.png}
	\caption{Current model: frequency of discount-factor monotonicity violations across time and pillar intervals.}
	\label{fig:finding1_df_monotonicity_current}
\end{figure}

\begin{figure}[H]
	\centering
	\includegraphics[width=0.85\textwidth]{hw_df_monotonicity_heatmap.png}
	\caption{HW1F benchmark: frequency of discount-factor monotonicity violations across time and pillar intervals.}
	\label{fig:finding1_df_monotonicity_hw}
\end{figure}

\paragraph{Kink index bands (cross-tenor smoothness).}
\begin{figure}[H]
	\centering
	\includegraphics[width=0.85\textwidth]{current_kink_bands.png}
	\caption{Current model: kink index bands over time (median and dispersion band).}
	\label{fig:finding1_kink_bands_current}
\end{figure}

\begin{figure}[H]
	\centering
	\includegraphics[width=0.85\textwidth]{hw_kink_bands.png}
	\caption{HW1F benchmark: kink index bands over time (median and dispersion band).}
	\label{fig:finding1_kink_bands_hw}
\end{figure}

\paragraph{Discount-factor wedge (multiplicative consistency).}
\begin{figure}[H]
	\centering
	\includegraphics[width=0.85\textwidth]{current_wedge_hist.png}
	\caption{Current model: wedge histogram for a representative $(T,u)$ pair.}
	\label{fig:finding1_wedge_hist_current}
\end{figure}

\begin{figure}[H]
	\centering
	\includegraphics[width=0.85\textwidth]{hw_wedge_hist.png}
	\caption{HW1F benchmark: wedge histogram for a representative $(T,u)$ pair.}
	\label{fig:finding1_wedge_hist_hw}
\end{figure}

\begin{figure}[H]
	\centering
	\includegraphics[width=0.85\textwidth]{current_wedge_vs_maturity.png}
	\caption{Current model: $p95(|\mathrm{wedge}|)$ vs maturity.}
	\label{fig:finding1_wedge_vs_maturity_current}
\end{figure}

\begin{figure}[H]
	\centering
	\includegraphics[width=0.85\textwidth]{hw_wedge_vs_maturity.png}
	\caption{HW1F benchmark: $p95(|\mathrm{wedge}|)$ vs maturity.}
	\label{fig:finding1_wedge_vs_maturity_hw}
\end{figure}

\subsection{Conclusion}
The diagnostics demonstrate that the IR simulation model does not enforce pathwise
arbitrage-free term-structure dynamics. Violations are systematic and maturity-dependent,
and are materially larger than those observed under an arbitrage-free benchmark.
This behaviour is a structural consequence of the model design and must be monitored,
benchmarked, and appropriately controlled when assessing long-horizon PFE metrics,
particularly for long-dated exposure profiles.
