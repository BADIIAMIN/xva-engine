\documentclass[11pt]{article}
\usepackage{amsmath}
\usepackage{amssymb}
\usepackage{algorithm}
\usepackage{algorithmic}
\usepackage{algpseudocode}
\usepackage{booktabs}
\usepackage{geometry}
 
\usepackage{graphicx}    % ← REQUIRED for \includegraphics
\usepackage{float}       % ← REQUIRED for [H]
\usepackage{caption}
\usepackage{subcaption}

\geometry{margin=1in}


% in main.tex or preamble
\graphicspath{{../outputs/}}


 

\title{Interest Rate Model Documentation\\
	{\large Comprehensive Model Description, Assumptions, and Limitations}}
\author{MUFG Securities EMEA plc}


\begin{document}
	
	\maketitle
	\tableofcontents
	\newpage
	
	%==============================================================================
	% MAIN SECTIONS
	%==============================================================================
	\section{Finding 1: Absence of Pathwise Arbitrage-Free Term-Structure Dynamics}
	\label{sec:finding_ir_arbitrage_free}
	
	\subsection{Finding Statement}
	
	The Interest Rate (IR) simulation model does not enforce pathwise arbitrage-free
	relationships across maturities.
	Each zero-rate maturity pillar is simulated as a distinct stochastic process,
	with cross-tenor dependence introduced exclusively via correlation between latent
	drivers.
	As a result, discount-factor multiplicative identities and forward-rate
	reconstruction conditions are not guaranteed to hold along individual Monte Carlo
	paths, particularly at long horizons or under elevated volatility regimes.
	
	This limitation may lead to economically implausible curve shapes and dynamic
	inconsistencies in simulated interest-rate term structures.
	
	\subsection{Model Mechanics Leading to the Finding}
	
	\subsubsection{Pillar-wise Stochastic Construction}
	
	In the implemented framework, each maturity pillar $k$ of the ultimate base curve
	is simulated independently through a shifted exponential Vasicek-type construction:
	
	\begin{equation}
		Y_k(t)
		=
		\bigl(g_k(t)+s_k\bigr)
		\exp\!\left(
		X_k(t) - \tfrac12 v_k^2(t)
		\right)
		-
		s_k,
	\end{equation}
	
	where:
	\begin{itemize}
		\item $X_k(t)$ follows an Ornstein--Uhlenbeck process,
		\item $g_k(t)$ is a deterministic mean function calibrated from the initial
		forward-forward curve,
		\item $s_k$ is a shift ensuring numerical stability,
		\item $v_k^2(t) = \mathrm{Var}[X_k(t)]$.
	\end{itemize}
	
	Each pillar has its own stochastic driver $X_k$; dependence across maturities is
	introduced solely through correlation of the Brownian motions driving $X_k$.
	
	\subsubsection{Absence of Term-Structure Arbitrage Constraints}
	
	The model does not impose:
	\begin{itemize}
		\item a short-rate representation,
		\item an instantaneous forward-rate framework,
		\item nor Heath--Jarrow--Morton (HJM) drift restrictions linking volatilities and
		drifts across maturities.
	\end{itemize}
	
	Consequently, while the model is marginally consistent with the initial yield
	curve in expectation, it does not guarantee joint consistency of the full curve
	along simulated paths.
	
	\subsection{Theoretical Corroboration}
	
	\subsubsection{Arbitrage-Free Requirement in Term-Structure Models}
	
	In arbitrage-free short-rate or HJM frameworks, discount factors satisfy the
	pathwise multiplicative identity:
	
	\begin{equation}
		DF(t,t+T)
		=
		DF(t,t+u)\cdot DF(t+u,t+T),
		\qquad \forall\ 0<u<T,
	\end{equation}
	
	since each discount factor is an exponential of the integral of a single
	underlying short rate over disjoint time intervals.
	
	Equivalently, instantaneous forward rates $f(t,T)$ must satisfy HJM drift
	restrictions of the form:
	
	\begin{equation}
		\alpha(t,T)
		=
		\sigma(t,T)\int_t^T \sigma(t,u)\,du,
	\end{equation}
	
	ensuring the absence of arbitrage under the risk-neutral measure.
	
	\subsubsection{Why the Implemented Model Violates These Identities}
	
	In the current model:
	\begin{itemize}
		\item $DF(t,t+T)$ is inferred from the simulated zero rate $Y_T(t)$,
		\item $DF(t+u,t+T)$ is inferred from an independently simulated curve at time $t+u$,
		\item no structural constraint enforces multiplicative consistency between the two.
	\end{itemize}
	
	As a result, discount factors reconstructed at different times along a path need
	not satisfy the multiplicative identity, even in the absence of interpolation
	effects.
	
	\subsection{Practical Implications for PFE}
	
	The absence of pathwise arbitrage-free dynamics may result in:
	\begin{itemize}
		\item Static inconsistencies, such as non-monotone discount factors or implausible
		forward-rate shapes at a given time point.
		\item Dynamic inconsistencies, where long-horizon discount factors do not match the
		compounded effect of shorter-dated discount factors along the same path.
		\item Potential bias in exposure profiles, particularly for long-dated interest
		rate derivatives whose valuation depends on the joint evolution of the full
		curve.
	\end{itemize}
	
	The materiality of these effects depends on portfolio maturity, volatility regime,
	and reliance on pathwise discounting.
	
	\subsection{Validation Tests and Diagnostics}
	
	\subsubsection{Test 1A: Static Discount-Factor Monotonicity}
	
	\paragraph{Objective}
	Verify the absence of static arbitrage at each simulation time by checking
	monotonicity of implied discount factors.
	
	\paragraph{Definition}
	For each path $p$, time $t_i$, and maturity pillar $M_k$:
	\begin{equation}
		DF(t_i,t_i+M_k) := \exp\!\bigl(-Y_k(t_i)\,M_k\bigr).
	\end{equation}
	
	The following condition should hold:
	\begin{equation}
		DF(t_i,t_i+M_{k+1}) \le DF(t_i,t_i+M_k),
		\qquad \forall k.
	\end{equation}
	
	\paragraph{Implementation}
	Discount factors are computed from simulated zero rates and violations are
	identified where discount factors increase with maturity.
	
	\paragraph{Outputs}
	\begin{itemize}
		\item Fraction of violated $(p,t,k)$ triplets.
		\item Maximum severity of monotonicity breaches.
		\item Heatmap of violation frequency by time and maturity interval.
	\end{itemize}
	
	\subsubsection{Test 1B: Forward-Rate Reconstruction Sanity}
	
	\paragraph{Objective}
	Assess whether implied forward rates between maturity pillars remain stable and
	economically plausible.
	
	\paragraph{Definition}
	For $M_i < M_j$, the implied forward rate is:
	\begin{equation}
		F_{i,j}(t)
		=
		\frac{Y_j(t)M_j - Y_i(t)M_i}{M_j - M_i}.
	\end{equation}
	
	\paragraph{Diagnostics}
	\begin{itemize}
		\item Distribution of forward rates by maturity bucket.
		\item Time evolution of percentile bands.
		\item Curvature metrics based on second finite differences.
	\end{itemize}
	
	\subsubsection{Test 1C: Pathwise Discount-Factor Multiplicative Identity}
	
	\paragraph{Objective}
	Directly quantify the failure of pathwise arbitrage-free discounting.
	
	\paragraph{Definition}
	For a simulation time step $u$ and maturity $T$:
	\begin{equation}
		\mathrm{Wedge}(t,u,T)
		=
		\log DF(t,t+T)
		-
		\log DF(t,t+u)
		-
		\log DF(t+u,t+T).
	\end{equation}
	
	In an arbitrage-free model, this quantity should be approximately zero.
	
	\paragraph{Implementation}
	Discount factors are inferred from simulated curves at times $t$ and $t+u$, and
	the wedge is computed pathwise.
	
	\paragraph{Outputs}
	\begin{itemize}
		\item Wedge distributions by maturity.
		\item Mean and percentile bands over time.
		\item Fraction of paths exceeding tolerance thresholds.
	\end{itemize}
	
	\subsection{Illustrations}
	
	The following figures are produced to support validation and governance:
	\begin{itemize}
		\item Discount-factor monotonicity violation heatmaps.
		\item Forward-rate curvature diagnostics across maturities.
		\item Wedge distribution histograms for long-dated maturities.
		\item Wedge magnitude as a function of maturity horizon.
	\end{itemize}
	
	\subsection{Conclusion}
	
	The absence of pathwise arbitrage-free term-structure dynamics is an inherent
	structural feature of the pillar-based IR simulation model and not an
	implementation defect.
	
	While the model remains suitable for PFE applications where marginal distributions
	dominate risk assessment, the limitation must be explicitly documented,
	quantitatively monitored, and benchmarked against arbitrage-free frameworks such
	as HJM or multi-factor short-rate models to assess materiality for specific
	portfolios.
	
	
	
	\subsection{Finding 1-Illustration --- Absence of Pathwise Arbitrage-Free Term-Structure Dynamics}
	\label{subsec:finding1_arbitrage_illustration}
	
	This section presents quantitative diagnostics and illustrative evidence supporting
	Finding~1, namely that the Interest Rate (IR) simulation model does not enforce
	pathwise arbitrage-free term-structure dynamics. The diagnostics focus on
	static no-arbitrage conditions, cross-tenor smoothness, and pathwise discount-factor
	consistency, and are benchmarked against an arbitrage-free Hull--White one-factor
	(HW1F) model.
	
	\medskip
	
	\subsubsection{Definition of the kink index (cross-tenor smoothness diagnostic)}
	
	Let the simulated zero-rate curve at simulation time $t$ be defined on a discrete set
	of maturity pillars $\{M_k\}_{k=0}^{K-1}$, with corresponding zero rates
	$\{Y_k(t)\}_{k=0}^{K-1}$.
	
	We define the first finite difference (local slope proxy) as:
	\[
	\Delta Y_k(t) = Y_{k+1}(t) - Y_k(t),
	\]
	
	and the second finite difference (local curvature proxy) as:
	\[
	\Delta^2 Y_k(t)
	=
	Y_{k+1}(t) - 2Y_k(t) + Y_{k-1}(t), \qquad k=1,\dots,K-2.
	\]
	
	The \emph{kink index} at time $t$ is defined as:
	\[
	\boxed{
		\mathrm{Kink}(t)
		=
		\max_{k=1,\dots,K-2}
		\left| \Delta^2 Y_k(t) \right|.
	}
	\]
	
	The kink index quantifies the maximum local curvature across adjacent maturity
	pillars. Large values indicate abrupt changes in slope between neighbouring tenors
	and reflect reduced cross-tenor coherence and economically implausible local
	deformations of the simulated term structure.
	
	\medskip
	
	\subsubsection{Static discount-factor monotonicity}
	
	\begin{figure}[H]
		\centering
		\includegraphics[width=0.85\textwidth]{finding1/current_df_monotonicity_heatmap.png}
		\caption{Current model: frequency of discount-factor monotonicity violations across
			time and maturity pillar intervals.}
		\label{fig:finding1_df_monotonicity_current}
	\end{figure}
	
	Figure~\ref{fig:finding1_df_monotonicity_current} reports the frequency with which the
	static no-arbitrage condition
	\[
	DF(t,T_{k+1}) \le DF(t,T_k)
	\]
	is violated across simulation time and adjacent maturity pillars under the current
	model. Violations are observed to increase both with simulation horizon and with
	maturity, reaching frequencies in excess of 40--50\% for some pillar intervals at
	longer horizons. This indicates that the simulated curve snapshots are not
	systematically decreasing in maturity and that static arbitrage conditions are not
	structurally enforced.
	
	\medskip
	
	\begin{figure}[H]
		\centering
		\includegraphics[width=0.85\textwidth]{finding1/hw_df_monotonicity_heatmap.png}
		\caption{HW1F benchmark: frequency of discount-factor monotonicity violations.}
		\label{fig:finding1_df_monotonicity_hw}
	\end{figure}
	
	In contrast, Figure~\ref{fig:finding1_df_monotonicity_hw} shows that under the HW1F
	benchmark model, monotonicity violations are rare, sporadic, and of limited magnitude.
	This behaviour is consistent with numerical discretisation effects rather than
	structural arbitrage.
	
	\medskip
	
	\subsubsection{Cross-tenor smoothness and kink diagnostics}
	
	\begin{figure}[H]
		\centering
		\includegraphics[width=0.85\textwidth]{finding1/current_kink_bands.png}
		\caption{Current model: kink index bands (median and 5th--95th percentiles) as a function
			of simulation time.}
		\label{fig:finding1_kink_current}
	\end{figure}
	
	Figure~\ref{fig:finding1_kink_current} reports the evolution of the kink index under the
	current model. The median kink index increases steadily with time, while the upper
	quantiles exhibit pronounced growth, reaching large values at longer horizons. The
	widening dispersion indicates that local curve smoothness deteriorates as stochastic
	variance accumulates, reflecting the absence of structural constraints linking
	adjacent maturity pillars.
	
	\medskip
	
	\begin{figure}[H]
		\centering
		\includegraphics[width=0.85\textwidth]{finding1/hw_kink_bands.png}
		\caption{HW1F benchmark: kink index bands.}
		\label{fig:finding1_kink_hw}
	\end{figure}
	
	Under the HW1F benchmark (Figure~\ref{fig:finding1_kink_hw}), the kink index remains
	low, stable, and tightly distributed over time. This reflects the fact that the entire
	curve is generated from a single latent short-rate factor, ensuring smooth and
	coherent cross-tenor dynamics by construction.
	
	\medskip
	
	\subsubsection{Pathwise discount-factor multiplicative consistency}
	
	\begin{figure}[H]
		\centering
		\includegraphics[width=0.85\textwidth]{finding1/current_wedge_hist.png}
		\caption{Current model: distribution of the pathwise discount-factor wedge for
			$T=20$ years and $u\approx 1$ month.}
		\label{fig:finding1_wedge_hist_current}
	\end{figure}
	
	Figure~\ref{fig:finding1_wedge_hist_current} displays the distribution of the
	pathwise discount-factor wedge:
	\[
	\log DF(t,T) - \log DF(t,u) - \log DF(t+u,T-u),
	\]
	under the current model. The distribution is wide and significantly dispersed away
	from zero, indicating substantial violations of the multiplicative discount-factor
	identity along individual simulation paths.
	
	\medskip
	
	\begin{figure}[H]
		\centering
		\includegraphics[width=0.85\textwidth]{finding1/hw_wedge_hist.png}
		\caption{HW1F benchmark: distribution of the pathwise discount-factor wedge.}
		\label{fig:finding1_wedge_hist_hw}
	\end{figure}
	
	By contrast, the HW1F benchmark (Figure~\ref{fig:finding1_wedge_hist_hw}) exhibits a
	tightly concentrated wedge distribution centred near zero, with magnitudes several
	orders of magnitude smaller, consistent with numerical approximation error only.
	
	\medskip
	
	\subsubsection{Wedge severity as a function of maturity}
	
	\begin{figure}[H]
		\centering
		\includegraphics[width=0.85\textwidth]{finding1/current_wedge_vs_maturity.png}
		\caption{Current model: 95th percentile of $|\text{wedge}|$ as a function of maturity.}
		\label{fig:finding1_wedge_maturity_current}
	\end{figure}
	
	Figure~\ref{fig:finding1_wedge_maturity_current} shows that the severity of pathwise
	discount-factor inconsistencies increases approximately linearly with maturity under
	the current model, reaching material levels for long-dated tenors. This indicates
	that arbitrage violations accumulate with horizon and disproportionately affect
	long-dated exposures.
	
	\medskip
	
	\begin{figure}[H]
		\centering
		\includegraphics[width=0.85\textwidth]{finding1/hw_wedge_vs_maturity.png}
		\caption{HW1F benchmark: 95th percentile of $|\text{wedge}|$ versus maturity.}
		\label{fig:finding1_wedge_maturity_hw}
	\end{figure}
	
	The HW1F benchmark exhibits uniformly low wedge magnitudes across maturities
	(Figure~\ref{fig:finding1_wedge_maturity_hw}), confirming that no structural
	amplification of inconsistencies occurs under an arbitrage-free short-rate framework.
	
	\medskip
	
	\subsubsection{Conclusion}
	
	The diagnostics presented above provide consistent and complementary evidence that
	the current IR simulation model does not enforce pathwise arbitrage-free
	term-structure dynamics. Violations are systematic, horizon-dependent, and materially
	larger than those observed under an arbitrage-free benchmark model. While this
	behaviour is a known and documented consequence of the model design, it represents a
	structural limitation that must be monitored, benchmarked, and appropriately
	controlled when assessing long-dated exposure profiles.
	
	
\end{document}